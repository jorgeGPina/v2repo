%% LyX 1.6.5 created this file.  For more info, see http://www.lyx.org/.
%% Do not edit unless you really know what you are doing.
\documentclass[11pt, spanish, es-tabla]{article}
\usepackage[spanish,es-tabla]{babel}
\usepackage{helvet}
\renewcommand{\familydefault}{\sfdefault}
\usepackage[T1]{fontenc}
\usepackage[utf8]{inputenc}
%\usepackage[latin9]{inputenc}
\usepackage{subfigure}
\usepackage{listings}
\usepackage{array}
\usepackage{textcomp}
\usepackage{graphics}
\usepackage{graphicx}
\usepackage{anysize}
\usepackage{url}
\usepackage[tikz]{bclogo}
\usepackage{booktabs}
\usepackage{enumitem}
\usepackage{pdflscape}



\makeatletter



%%%%%%%%%%%%%%%%%%%%%%%%%%%%%% LyX specific LaTeX commands.
%% Because html converters don't know tabularnewline
\providecommand{\tabularnewline}{\\}

%%%%%%%%%%%%%%%%%%%%%%%%%%%%%% Textclass specific LaTeX commands.
\newenvironment{lyxcode}
{\par\begin{list}{}{
\setlength{\rightmargin}{\leftmargin}
\setlength{\listparindent}{0pt}% needed for AMS classes
\raggedright
\setlength{\itemsep}{0pt}
\setlength{\parsep}{0pt}
\normalfont\ttfamily}%
 \item[]}
{\end{list}}

\makeatother

\usepackage{babel}
\addto\shorthandsspanish{\spanishdeactivate{~<>}}

\marginsize{3cm}{3cm}{2.75cm}{2.5cm}
\definecolor{mycolor}{RGB}{255, 245, 213}

\renewcommand{\lstlistingname}{Código}% Listing -> Código



\begin{document}
\begin{center}
{\Large \thispagestyle{empty}}{\large UNIVERSIDAD DE ALCALÁ}{\Large{} }
\par\end{center}{\Large \par}

\begin{center}
Departamento de Automática
\par\end{center}

\begin{center}
Grado en Ingeniería Informática

\par\end{center}

\vspace{6cm}

\begin{center}
{\LARGE Instrucciones para la práctica 2}
\par\end{center}{\LARGE \par}
\vspace{11cm}


\begin{center}
{\large Sistemas Operativos}
\par\end{center}{\large \par}


\newpage{}
$\ $
\thispagestyle{empty} % para que no se numere esta pagina
\newpage{}
\tableofcontents{}
\newpage{}
\marginsize{3cm}{3cm}{1cm}{2.5cm}


\section{Antes de comenzar la práctica}
Esta práctica se realizará, en general, en \textbf{grupos de cuatro estudiantes}. Se valorará con el profesor posibles excepciones, de acuerdo a situaciones particulares, siempre sin exceder el grupo de 4 alumnos. Asimismo,  \textbf{los  miembros de cada grupo deben pertenecer al mismo horario de laboratorio} salvo motivos justificados que también estimará el profesor.

\subsection{Tarea 1 - Organización}

La primera tarea de los estudiantes para esta práctica será determinar cuáles serán sus respectivas responsabilidades en cada una de las siete fases obligatorias de la práctica. Cada estudiante puede desempeñar \textbf{en cada fase} uno de los siguientes roles: 

\begin{itemize}
	\item \textbf{Líder (L)}: Será el responsable final de que la fase se realice correctamente. Esto incluye, entre otras tareas:
	\begin{enumerate}
		\item Garantizar que se cumplen todas las funcionalidades solicitadas en la fase.
		\item Garantizar que la codificación realizada sigue unos criterios de calidad adecuados.
		\item Garantizar que se han realizado todas las pruebas pertinentes para evaluar la fase.
		\item Garantizar que el código implementado se encuentra correctamente documentado.
		\item Gestionar cualquier conflicto que surja entre los integrantes del grupo. 
		\item Tomar las decisiones pertinentes cuando no haya acuerdo entre diversos miembros del grupo. 
		\item Planificar la temporización de la fase. 
	\end{enumerate}	  
		
	\item \textbf{Programador (P)}: Será el responsable final de codificar las funcionalidades solicitadas en la fase. Esto incluye, entre otras tareas:
	\begin{enumerate}
		\item Seguir el estilo de programación acordado por todos los integrantes del grupo. 
		\item Implementar las funcionalidades solicitadas. 
		\item Realizar las correcciones pertinentes respecto a la programación detectadas y notificadas por cualquier otro usuario. 
	\end{enumerate}
	
	\item \textbf{Evaluador (E)}: Será el responsable final de testear las funcionalidades solicitadas en la fase. Esto incluye, entre otras tareas: 
	\begin{enumerate}
		\item Determinar los casos de prueba que se deben a llevar a cabo en la fase. 
		\item Llevar a cabo dichas pruebas.
		\item Avisar al programador y al líder si se detecta cualquier error en la programación. 
		\item Realizar los casos de prueba no implementados detectados y notificados por cualquier otro usuario.
	\end{enumerate}
	
	\item \textbf{Documentalista (D)}: Será el responsable final de que el código quede correctamente comentado. Entre sus tareas, se incluyen: 
	\begin{enumerate}
		\item Documentar los fragmentos de código relevantes incluidos por el programador tanto en las fases obligatorias como en cualquier añadido extra.   
		\item Documentar aquellos fragmentos de código cuya comprensión pueda resultar compleja. 
		\item Avisar al programador y al líder si se detecta cualquier error en la programación (fragmentos no empleados, diferentes estilos de programación, etc.).  
	\end{enumerate}
\end{itemize}

Las fases de la práctica son las siguientes: 

\begin{enumerate}
	\item Ciclo de ejecución de órdenes.
	\item Ejecución de órdenes externas simples en primer plano.
	\item Ejecución de órdenes externas simples en segundo plano.
	\item Realización de \textit{makefile}.
	\item Ejecución de secuencia de órdenes (sólo \textbf{obligatorio} órdenes  externas).
	\item Tratamiento de redirecciones.
	\item Implementación de tuberías o \textit{pipes}.	
\end{enumerate}

Para determinar las tareas de cada alumno, a continuación se ilustran las diferentes posibilidades en las Tablas \ref{table:eleccionActiv1}, \ref{table:eleccionActiv2} y \ref{table:eleccionActiv3}. En caso de que dos o más alumnos no se pongan de acuerdo sobre su rol a lo largo de la actividad, se determinará de manera aleatoria. 

\begin{table}[h!]
\centering
\caption{División de tareas por usuario (4 alumnos).} 
\label{table:eleccionActiv1}
\begin{tabular}{@{}cccccccc@{}}
\toprule
\textbf{USUARIO}   & \textbf{FASE 1} & \textbf{FASE 2} & \textbf{FASE 3} & \textbf{FASE 4} & \textbf{FASE 5} & \textbf{FASE 6} & \textbf{FASE 7} \\ \midrule
\textbf{ALUMNO 1} &      
P           &                 
E			&             
D			&                 
L			&                 
D			&                  
P			&    
E           \\
\textbf{ALUMNO 2} &      
L           &     
P           &  
P           &                 
E			&                 
E			&                  
L			&                 
D			\\
\textbf{ALUMNO 3} &      
D           &                 
L			&                 
E			&   
P           &  
P           &                  
D			&                 
L			\\
\textbf{ALUMNO 4} &      
E           &                 
D			&                 
L			&                 
D			&                 
L			&                  
E			&     
P            \\ \bottomrule
\end{tabular}
\end{table}

\vspace{-0.5cm}

\begin{table}[h!]
\centering
\caption{División de tareas por usuario (3 alumnos).} 
\label{table:eleccionActiv2}
\begin{tabular}{@{}cccccccc@{}}
\toprule
\textbf{USUARIO}   & \textbf{FASE 1} & \textbf{FASE 2} & \textbf{FASE 3} & \textbf{FASE 4} & \textbf{FASE 5} & \textbf{FASE 6} & \textbf{FASE 7} \\ \midrule
\textbf{ALUMNO 1} &      
L y D       &                 
P			&             
E			&                 
L y D      	&                 
E			&                  
L y D       &    
P           \\
\textbf{ALUMNO 2} &      
P           &     
E           &  
L y D       &                 
P			&                 
L y D      	&                  
P			&                 
E			\\
\textbf{ALUMNO 3} &      
E           &                 
L y D       &                 
P			&   
E 	        &  
P           &                  
E  		    &                 
L y D			\\
\bottomrule
\end{tabular}
\end{table}

\begin{table}[h!]
\centering
\caption{División de tareas por usuario (2 alumnos).} 
\label{table:eleccionActiv3}
\begin{tabular}{@{}cccccccc@{}}
\toprule
\textbf{USUARIO}   & \textbf{FASE 1} & \textbf{FASE 2} & \textbf{FASE 3} & \textbf{FASE 4} & \textbf{FASE 5} & \textbf{FASE 6} & \textbf{FASE 7} \\ \midrule
\textbf{ALUMNO 1} &      
P y E       &                 
E y L    	&             
L y P		&                 
D y P    	&                 
D y E		&                  
P y L 	    &    
D y E       \\
\textbf{ALUMNO 2} &      
L y D       &     
D y P       &  
D y E  		&                 
L y E		&                 
P y L  		&                  
D y E		&                 
P y L		\\ \bottomrule
\end{tabular}
\end{table}

\subsection{Tarea 2 - Planificación}

Los estudiantes deberán comunicar a los profesores quienes serán los integrantes del grupo así como el rol que desempeñará cada alumno en cada fase en un archivo denominado \textbf{practica2.odt} cuyo contenido se especifica en el enlace de envío de la práctica en Blackboard.\\

\textbf{La práctica debe desarrollarse obligatoriamente usando un repositorio GitHub}.

~

\begin{bclogo}[couleur = green!20!blue!20,arrondi =0.1, logo = \bcattention, ombre = false]{\textbf{¡IMPORTANTE!}}
\noindent\textbf{LOS ESTUDIANTES DEBEN SER CAPACES DE COMPRENDER Y APLICAR \underline{TODAS LAS TAREAS} DE CUALQUIER FASE \underline{INDEPENDIENTEMENTE DE SU} \underline{ROL} EN CADA FASE; CUALQUIER TAREA DE LA PRÁCTICA PUEDE SER OBJETO DE PREGUNTA EN LAS DIFERENTES PRUEBAS DE EVALUACIÓN.} 
\end{bclogo}

\vspace{-0.5cm}
\section{Durante la realización de la práctica}

La práctica tiene una duración estimada de cuatro sesiones presenciales en el laboratorio en las que deben realizarse las siete fases que componen la práctica. En cada sesión, el profesor del laboratorio proporcionará inicialmente los detalles oportunos sobre el desarrollo de cada fase.


\section{Tras finalizar la práctica}

Una vez finalizada la práctica y subida a la plataforma Blackboard siguiendo exactamente las indicaciones del enlace asociado, los estudiantes deben realizar una breve prueba individual de evaluación de la práctica (\textbf{$PL_{2}$}) en la plataforma (a modo de defensa) respondiendo a preguntas cortas (prueba de tipo ensayo) acerca de algunos aspectos del desarrollo de la misma. 

\subsection{Tarea 3 - Prueba individual}

La prueba $PL_{2}$ se realizará en horario de laboratorio, en el grupo al que pertenezca cada alumno. La  fecha y  hora, así como el formato de la prueba, como las restantes, se detallará tanto en la plataforma Blackboard como en el propio laboratorio, el primer día de realización de la práctica. Los alumnos pueden  consultar la fecha de esta prueba en la página siguiente  de la  Escuela Politécnica Superior: \url{http://escuelapolitecnica.uah.es/estudiantes/gcalendar-2GII.asp}. Cabe recordar que estas preguntas pueden estar relacionadas con alguna fase que ha realizado un compañero por lo que es imprescindible comprender toda la práctica desde todos los roles.  


%\subsection{Tarea 5 - Coevaluación/autoevaluación}
%
%En cuanto a la coevaluación(autoevaluación es importante que se responda de manera ética y justificada cuál ha sido la calidad del trabajo desempeñado por el resto de compañeros así como el propio trabajo. Esta información pretende ser empleada por los profesores de la asignatura junto con la defensa para detectar que estudiantes han trabajado más en la práctica. Dicha información, anónima para el resto de compañeros, será empleada para otorgar a los alumnos con mejores puntuaciones un logro\footnote{Se informará en el aula acerca de este instrumento de evaluación adicional.} de la asignatura en la plataforma \textit{Blackboard}. Además, será parte de la nota de la práctica.
%
%~
%
%Las preguntas de coevaluación/autoevaluación serán las siguientes: 
%
%\begin{enumerate}
%	\item Evalúe la asistencia y puntualidad mostrado por el compañer@ durante la práctica.
%	\begin{small}
%	
%	\begin{enumerate}[label=(\alph*)]
%		\item Ha asistido a todas las reuniones del grupo y lo ha hecho de manera puntual.
%		\item Ha asistido a todas las reuniones del grupo y lo ha hecho, casi siempre, de manera puntual. 
%		\item Ha asistido habitualmente a las reuniones del grupo y lo ha hecho, casi siempre, de manera puntual. 
%		\item Ha asistido a menos de la mitad de las reuniones del grupo y/o ha llegado con poca puntualidad.
%		\item No ha asistido a ninguna de las reuniones del grupo y/o ha llegado considerablemente tarde.
%	\end{enumerate}
%	\end{small}
%
%	\item Evalúe la responsabilidad y perseverancia mostrado por el compañer@ durante la práctica. 
%\begin{small}
%	\begin{enumerate}[label=(\alph*)]
%		\item Ha cumplido siempre con sus responsabilidades en la práctica aunque lo haga fuera del plazo establecido por los miembros del grupo.
%		\item Suele haber cumplido con sus responsabilidades en la práctica aunque lo haga fuera del plazo establecido por los miembros del grupo. 
%		\item Suele haber cumplido con sus responsabilidades en la práctica pero ha abandonado alguna actividad si no ha obtenido resultados en el plazo establecido por los miembros del grupo. 
%		\item Rara vez ha cumplido con sus responsabilidades en la práctica y/o ha abandonado alguna actividad si no ha obtenido resultados en un breve periodo de tiempo.
%		\item No ha cumplido nunca con sus responsabilidades en la práctica y/o  ha abandonado las actividades al más mínimo problema.
%	\end{enumerate}
%\end{small}
%
%	\item Evalúe la capacidad de colaboración de este compañer@ con el resto de los miembros del grupo en el desarrollo de la práctica. 
%\begin{small}
%	\begin{enumerate}[label=(\alph*)]
%		\item Ha colaborado con los miembros del grupo en todas las tareas de la práctica.
%		\item Ha colaborado con el resto de los miembros del grupo en la mayoría de las tareas de la práctica. 
%		\item Ha colaborado con el resto de los miembros del grupo en las tareas sencillas de la práctica. 
%		\item Rara vez ha colaborado con el resto de los miembros del grupo en las tareas de la práctica.
%		\item No ha colaborado nunca con el resto de los miembros del grupo en las tareas de la práctica.
%	\end{enumerate}
%\end{small}
%
%	\item Evalúe la organización y el liderazgo mostrado por el compañer@ durante la práctica. 
%\begin{small}
%	\begin{enumerate}[label=(\alph*)]
%		\item Ha mostrado grandes dotes de liderazgo y organización.
%		\item Ha contribuido ampliamente en la organización de la práctica. 
%		\item Ha participado regularmente en la organización de la práctica.
%		\item Ha participado ligeramente en la organización de la práctica.
%		\item No ha participado en la organización de la práctica.
%	\end{enumerate}
%\end{small}
%
%	\item Evalúe la capacidad de resolución de conflictos mostrado por el compañer@ durante la práctica. 
%\begin{small}
%	\begin{enumerate}[label=(\alph*)]
%		\item Busca y sugiere soluciones a los problemas.
%		\item A veces sugiere ideas o soluciones y siempre ayuda en refinar soluciones sugeridas por otros.  
%		\item No sugiere o refina soluciones, pera está dispuesto a tratar soluciones propuestas por otros.
%		\item No sugiere o refina soluciones. A veces está dispuesto a tratar soluciones propuestas por otros.
%		\item No trata de resolver problemas o ayudar a otros a resolverlos. Deja a otros hacer el trabajo.
%	\end{enumerate}
%\end{small}
%
%	\item Justifique brevemente los resultados otorgados al compañer@ en esta práctica.
%
%\end{enumerate}
%
%\subsection{Tarea 6 - Encuesta}
%
%Además, el alumno puede rellenar una encuesta denominada \textit{Danos tu opinión} que se encuentra en la plataforma \textit{Blackboard}. Es importante recordar que, de cara a la evaluación de la asignatura, la actitud mostrada a lo largo del curso por los estudiantes puede ayudar a los profesores a redondear la nota final de un alumno (siempre y cuando el alumno haya superado satisfactoriamente la asignatura-asignatura aprobada-).

\section{Evaluación}

La práctica representa el 15\% de la nota de la asignatura, es decir 1.5 puntos de la nota final. Dicha nota está formada por 3 apartados diferentes: 

	\begin{itemize}
		\item Entrega de la práctica (prueba grupal: \textbf{$E_{2}$}); común a todos los miembros del grupo): \textbf{0.5 puntos}. Esta prueba será evaluada mediante una rúbrica,  \textit{RúbricaPE2-práctica2}, que puede consultarse en la plataforma Blackboard. Los archivos que incluirá son detallados en un enlace de envío.
		%\item Coevaluación/autoevaluación (individual para todos los alumnos): 0.33 puntos.
		\item Prueba de  tipo ensayo sobre los contenidos de la práctica (prueba individual: \textbf{$PL_{2}$}): \textbf{0,7 puntos}.
		\item Evaluación del uso de GitHub de cada alumno (prueba individual: \textbf{$Git_{2}$}): \textbf{0.3 puntos}.
	\end{itemize}


%El primer apartado (entrega) se detalla en la rúbrica que aparece en Blackboard. Los apartados restantes (coevaluación/autoevaluación, defensa y diario) se muestran en la siguiente rúbrica (véase Figura \ref{fig:rubrica}).

%\clearpage
%
%\begin{landscape}
%
%\begin{figure}[ht!]
%\centering
%\includegraphics[width=25cm]{rubrica2}
%\caption{Rúbrica de defensa, coevaluación/autoevaluación y diario.}
%\label{fig:rubrica}
%\end{figure}
%
%\end{landscape}

%
%\section{Información adicional}
%
%\noindent\textbf{¿Por qué está práctica está en Blackboard y en Moodle? 
%}
%
%~
%
%\noindent Dada la relevancia de los contenidos que abarca la práctica 3 en el contexto de la asignatura de Sistemas Operativos, la principal meta que se persigue con este modo de evaluación, a través de otra plataforma complementaria a Moodle, consiste en validar si este tipo de prueba por grupos permite lograr mejores rendimientos de los alumnos así como proporcionarles un \textit{feedback} más completo. Si los resultados obtenidos son los previsibles, se espera extender este tipo de prueba a todos los grupos reducidos del grado de Ingeniería Informática, u otros, en futuros cursos académicos. 
\end{document}